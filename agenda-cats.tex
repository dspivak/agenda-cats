\documentclass[11pt, one side, article]{memoir}


\settrims{0pt}{0pt} % page and stock same size
\settypeblocksize{*}{34.5pc}{*} % {height}{width}{ratio}
\setlrmargins{*}{*}{1} % {spine}{edge}{ratio}
\setulmarginsandblock{.98in}{.98in}{*} % height of typeblock computed
\setheadfoot{\onelineskip}{2\onelineskip} % {headheight}{footskip}
\setheaderspaces{*}{1.5\onelineskip}{*} % {headdrop}{headsep}{ratio}
\checkandfixthelayout


\usepackage{amsthm}
\usepackage{mathtools}

\usepackage[inline]{enumitem}
\usepackage{ifthen}
\usepackage[utf8]{inputenc} %allows non-ascii in bib file
\usepackage{xcolor}

\usepackage[backend=biber, backref=true, maxbibnames = 10, style = alphabetic]{biblatex}
\usepackage[bookmarks=true, colorlinks=true, linkcolor=blue!50!black,
citecolor=orange!50!black, urlcolor=orange!50!black, pdfencoding=unicode]{hyperref}
\usepackage[capitalize]{cleveref}

\usepackage{tikz}

\usepackage{amssymb}
\usepackage{newpxtext}
\usepackage[varg,bigdelims]{newpxmath}
\usepackage{mathrsfs}
\usepackage{dutchcal}
\usepackage{fontawesome}



% cleveref %
  \newcommand{\creflastconjunction}{, and\nobreakspace} % serial comma
  \crefformat{enumi}{\##2#1#3}
  \crefalias{chapter}{section}


% biblatex %
  \addbibresource{Library20220706.bib} 

% hyperref %
  \hypersetup{final}

% enumitem %
  \setlist{nosep}
  \setlistdepth{6}



% tikz %



  \usetikzlibrary{ 
  	cd,
  	math,
  	decorations.markings,
		decorations.pathreplacing,
  	positioning,
  	arrows.meta,
  	shapes,
		shadows,
		shadings,
  	calc,
  	fit,
  	quotes,
  	intersections,
    circuits,
    circuits.ee.IEC
  }
  
  \tikzset{
biml/.tip={Glyph[glyph math command=triangleleft, glyph length=.95ex]},
bimr/.tip={Glyph[glyph math command=triangleright, glyph length=.95ex]},
}

\tikzset{
	tick/.style={postaction={
  	decorate,
    decoration={markings, mark=at position 0.5 with
    	{\draw[-] (0,.4ex) -- (0,-.4ex);}}}
  }
} 
\tikzset{
	slash/.style={postaction={
  	decorate,
    decoration={markings, mark=at position 0.5 with
    	{\draw[-] (.3ex,.3ex) -- (-.3ex,-.3ex);}}}
  }
} 

\newcommand{\bito}[1][]{
	\begin{tikzcd}[ampersand replacement=\&, cramped]\ar[r, biml-bimr, "#1"]\&~\end{tikzcd}  
}
\newcommand{\bifrom}[1][]{
	\begin{tikzcd}[ampersand replacement=\&, cramped]\ar[r, bimr-biml, "{#1}"]\&~\end{tikzcd}  
}
\newcommand{\bifromlong}[2][]{
	\begin{tikzcd}[ampersand replacement=\&, column sep=#2, cramped]\ar[r, bimr-biml, "#1"]\&~\end{tikzcd}  
}

% Adjunctions
\newcommand{\adj}[5][30pt]{%[size] Cat L, Left, Right, Cat R.
\begin{tikzcd}[ampersand replacement=\&, column sep=#1]
  #2\ar[r, shift left=7pt, "#3"]
  \ar[r, phantom, "\scriptstyle\Rightarrow"]\&
  #5\ar[l, shift left=7pt, "#4"]
\end{tikzcd}
}

\newcommand{\adjr}[5][30pt]{%[size] Cat R, Right, Left, Cat L.
\begin{tikzcd}[ampersand replacement=\&, column sep=#1]
  #2\ar[r, shift left=7pt, "#3"]\&
  #5\ar[l, shift left=7pt, "#4"]
  \ar[l, phantom, "\scriptstyle\Leftarrow"]
\end{tikzcd}
}

\newcommand{\xtickar}[1]{\begin{tikzcd}[baseline=-0.5ex,cramped,sep=small,ampersand 
replacement=\&]{}\ar[r,tick, "{#1}"]\&{}\end{tikzcd}}
\newcommand{\xslashar}[1]{\begin{tikzcd}[baseline=-0.5ex,cramped,sep=small,ampersand 
replacement=\&]{}\ar[r,tick, "{#1}"]\&{}\end{tikzcd}}



  
  % amsthm %
\theoremstyle{definition}
\newtheorem{definitionx}{Definition}[chapter]
\newtheorem{examplex}[definitionx]{Example}
\newtheorem{remarkx}[definitionx]{Remark}
\newtheorem{notation}[definitionx]{Notation}


\theoremstyle{plain}

\newtheorem{theorem}[definitionx]{Theorem}
\newtheorem{proposition}[definitionx]{Proposition}
\newtheorem{corollary}[definitionx]{Corollary}
\newtheorem{lemma}[definitionx]{Lemma}
\newtheorem{warning}[definitionx]{Warning}
\newtheorem*{theorem*}{Theorem}
\newtheorem*{proposition*}{Proposition}
\newtheorem*{corollary*}{Corollary}
\newtheorem*{lemma*}{Lemma}
\newtheorem*{warning*}{Warning}
%\theoremstyle{definition}
%\newtheorem{definition}[theorem]{Definition}
%\newtheorem{construction}[theorem]{Construction}

\newenvironment{example}
  {\pushQED{\qed}\renewcommand{\qedsymbol}{$\lozenge$}\examplex}
  {\popQED\endexamplex}
  
 \newenvironment{remark}
  {\pushQED{\qed}\renewcommand{\qedsymbol}{$\lozenge$}\remarkx}
  {\popQED\endremarkx}
  
  \newenvironment{definition}
  {\pushQED{\qed}\renewcommand{\qedsymbol}{$\lozenge$}\definitionx}
  {\popQED\enddefinitionx} 

    
%-------- Single symbols --------%
	
\DeclareSymbolFont{stmry}{U}{stmry}{m}{n}
\DeclareMathSymbol\fatsemi\mathop{stmry}{"23}

\DeclareFontFamily{U}{mathx}{\hyphenchar\font45}
\DeclareFontShape{U}{mathx}{m}{n}{
      <5> <6> <7> <8> <9> <10>
      <10.95> <12> <14.4> <17.28> <20.74> <24.88>
      mathx10
      }{}
\DeclareSymbolFont{mathx}{U}{mathx}{m}{n}
\DeclareFontSubstitution{U}{mathx}{m}{n}
\DeclareMathAccent{\widecheck}{0}{mathx}{"71}


%-------- Renewed commands --------%

\renewcommand{\ss}{\subseteq}

%-------- Other Macros --------%


\DeclarePairedDelimiter{\present}{\langle}{\rangle}
\DeclarePairedDelimiter{\copair}{[}{]}
\DeclarePairedDelimiter{\floor}{\lfloor}{\rfloor}
\DeclarePairedDelimiter{\ceil}{\lceil}{\rceil}
\DeclarePairedDelimiter{\corners}{\ulcorner}{\urcorner}
\DeclarePairedDelimiter{\ihom}{[}{]}

\DeclareMathOperator{\Hom}{Hom}
\DeclareMathOperator{\Mor}{Mor}
\DeclareMathOperator{\dom}{dom}
\DeclareMathOperator{\cod}{cod}
\DeclareMathOperator{\idy}{idy}
\DeclareMathOperator{\comp}{com}
\DeclareMathOperator*{\colim}{colim}
\DeclareMathOperator{\im}{im}
\DeclareMathOperator{\ob}{Ob}
\DeclareMathOperator{\Tr}{Tr}
\DeclareMathOperator{\el}{El}




\newcommand{\const}[1]{\texttt{#1}}%a constant, or named element of a set
\newcommand{\Set}[1]{\mathsf{#1}}%a named set
\newcommand{\ord}[1]{\mathsf{#1}}%an ordinal
\newcommand{\cat}[1]{\mathcal{#1}}%a generic category
\newcommand{\Cat}[1]{\mathbf{#1}}%a named category
\newcommand{\fun}[1]{\mathrm{#1}}%a function
\newcommand{\Fun}[1]{\mathrm{#1}}%a named functor




\newcommand{\id}{\mathrm{id}}
\newcommand{\then}{\mathbin{\fatsemi}}

\newcommand{\cocolon}{:\!}


\newcommand{\iso}{\cong}
\newcommand{\too}{\longrightarrow}
\newcommand{\tto}{\rightrightarrows}
\newcommand{\To}[2][]{\xrightarrow[#1]{#2}}
\renewcommand{\Mapsto}[1]{\xmapsto{#1}}
\newcommand{\Tto}[3][13pt]{\begin{tikzcd}[sep=#1, cramped, ampersand replacement=\&, text height=1ex, text depth=.3ex]\ar[r, shift left=2pt, "#2"]\ar[r, shift right=2pt, "#3"']\&{}\end{tikzcd}}
\newcommand{\Too}[1]{\xrightarrow{\;\;#1\;\;}}
\newcommand{\from}{\leftarrow}
\newcommand{\ffrom}{\leftleftarrows}
\newcommand{\From}[1]{\xleftarrow{#1}}
\newcommand{\Fromm}[1]{\xleftarrow{\;\;#1\;\;}}
\newcommand{\surj}{\twoheadrightarrow}
\newcommand{\inj}{\rightarrowtail}
\newcommand{\wavyto}{\rightsquigarrow}
\newcommand{\lollipop}{\multimap}
\newcommand{\imp}{\Rightarrow}
\renewcommand{\iff}{\Leftrightarrow}
\newcommand{\down}{\mathbin{\downarrow}}
\newcommand{\fromto}{\leftrightarrows}
\newcommand{\tickar}{\xtickar{}}
\newcommand{\slashar}{\xslashar{}}



\newcommand{\inv}{^{-1}}
\newcommand{\op}{^\tn{op}}
\newcommand{\co}{^\tn{co}}

\newcommand{\tn}[1]{\textnormal{#1}}
\newcommand{\ol}[1]{\overline{#1}}
\newcommand{\ul}[1]{\underline{#1}}
\newcommand{\wt}[1]{\widetilde{#1}}
\newcommand{\wh}[1]{\widehat{#1}}
\newcommand{\wc}[1]{\widecheck{#1}}
\newcommand{\ubar}[1]{\underaccent{\bar}{#1}}

\newcommand{\dual}[1]{{#1}^\vee}



\newcommand{\bb}{\mathbb{B}}
\newcommand{\cc}{\mathbb{C}}
\newcommand{\nn}{\mathbb{N}}
\newcommand{\pp}{\mathbb{P}}
\newcommand{\qq}{\mathbb{Q}}
\newcommand{\zz}{\mathbb{Z}}
\newcommand{\rr}{\mathbb{R}}


\newcommand{\finset}{\Cat{Fin}}
\newcommand{\smset}{\Cat{Set}}
\newcommand{\smcat}{\mathbb{C}\Cat{at}}
\newcommand{\catsharp}{\Cat{Cat}^{\sharp}}
\newcommand{\ppolyfun}{\mathbb{P}\Cat{olyFun}}
\newcommand{\ccatsharp}{\mathbb{C}\Cat{at}^{\sharp}}
\newcommand{\ccatsharpdisc}{\mathbb{C}\Cat{at}^{\sharp}_{\tn{disc}}}
\newcommand{\ccatsharplin}{\mathbb{C}\Cat{at}^{\sharp}_{\tn{lin}}}
\newcommand{\ccatsharpdisccon}{\mathbb{C}\Cat{at}^{\sharp}_{\tn{disc,con}}}
\newcommand{\sspan}{\mathbb{S}\Cat{pan}}

\newcommand{\List}{\Fun{List}}
\newcommand{\set}{\tn{-}\Cat{Set}}




\newcommand{\yon}{\mathcal{y}}
\newcommand{\poly}{\Cat{Poly}}
%\newcommand{\poly}{\smset[\yon]}
\newcommand{\ppoly}{\mathbb{P}\Cat{oly}}
\newcommand{\0}{\textsf{0}}
\newcommand{\1}{\tn{\textsf{1}}}
\newcommand{\2}{\tn{\textsf{2}}}
\newcommand{\3}{\tn{\textsf{3}}}
\newcommand{\4}{\tn{\textsf{4}}}
\newcommand{\5}{\tn{\textsf{5}}}
\newcommand{\6}{\tn{\textsf{6}}}
\newcommand{\7}{\tn{\textsf{7}}}
\newcommand{\8}{\tn{\textsf{8}}}
\newcommand{\9}{\tn{\textsf{9}}}
\newcommand{\tri}{\mathbin{\triangleleft}}
\newcommand{\triright}{\mathbin{\triangleright}}
\newcommand{\tripow}[1]{^{\tri #1}}
\newcommand{\rdag}{^{\rotatebox{0}{$\dagger$}}}
\newcommand{\ldag}{^{\rotatebox{180}{$\dagger$}}}
\newcommand{\ot}[2]{\,\mbox{${}_{#1}\otimes_{#2}$}\,}
\newcommand{\ih}[4]{{}_{#1}[#2,#3]_{#4}}

\newcommand{\fd}{\Cat{FD}}
\newcommand{\mn}{\Cat{Mn}}
\newcommand{\ddial}{\Cat{DDial}}


% lenses
\newcommand{\biglens}[2]{
     \begin{bmatrix}{\vphantom{f_f^f}#2} \\ {\vphantom{f_f^f}#1} \end{bmatrix}
}
\newcommand{\littlelens}[2]{
     \begin{bsmallmatrix}{\vphantom{f}#2} \\ {\vphantom{f}#1} \end{bsmallmatrix}
}
\newcommand{\lens}[2]{
  \relax\if@display
     \biglens{#1}{#2}
  \else
     \littlelens{#1}{#2}
  \fi
}



\newcommand{\qand}{\quad\text{and}\quad}
\newcommand{\qqand}{\qquad\text{and}\qquad}


\newcommand{\coto}{\nrightarrow}
\newcommand{\cofun}{{\raisebox{2pt}{\resizebox{2.5pt}{2.5pt}{$\setminus$}}}}

\newcommand{\coalg}{\tn{-}\Cat{Coalg}}

\newcommand{\bic}[2]{{}_{#1}\Cat{Comod}_{#2}}

% ---- Changeable document parameters ---- %

\linespread{1.1}
%\allowdisplaybreaks
\setsecnumdepth{chapter}
\settocdepth{chapter}
\setlength{\parindent}{15pt}
\setcounter{tocdepth}{1}



%--------------- Document ---------------%
\begin{document}

\title{The root source of structures\\on $\poly$ and dependent dialectica
}

\author{David I. Spivak}

\date{\vspace{-.2in}}

\maketitle

\begin{abstract}
Both the category of polynomial functors and the dependent Dialectica category are extremely rich, and for the same reason: each emerges as the free distributive category on a simpler category, namely on the free object and the free arrow, respectively: $\poly=\fd(\bullet)$, $\ddial=\fd(\bullet\to\bullet)$. But in fact, the free distributive category construction has a kind of "square root": a simpler construction, which I'll call the \emph{menu category} construction, with the property that $\fd(\cat{C})\cong\mn(\mn(\cat{C}))$.

The menu category $\mn(\cat{C})$ is given by the free coproduct completion of $\cat{C}\op$ and as such enjoys excellent formal properties: $\mn(\cat{C})$ always has extensive coproducts; $\mn(\cat{C})$ has a distributive monoidal operation for every monoidal operation on $\cat{C}$; $\mn(\cat{C})$ is complete/cocomplete if $\cat{C}$ is cocomplete/complete; $\mn(\cat{C})$has monoidal closures/coclosures if $\cat{C}$ has monoidal coclosures/closures, etc. We can also define a substitution operation on $\fd(\cat{C})$ for any monoidal category $\cat{C}$. 

From this laundry list of formal properties and the fact that the free object category $\bullet$ has exactly one of every categorical structure, we recover many known properties of $\poly$, $\Cat{DDial}$, etc. Focusing on $\poly$, we recover the usual operations $+$, $\times$, $\otimes$, $\tri$, as well as many if not all the known facts about $\times$ and $\otimes$ being monoidal closed, $\otimes$ and $\tri$ being coclosed, etc.  \\

\smallskip

\noindent
\textbf{Keywords:} **
\end{abstract}

%-------- Chapter --------%
\chapter{Introduction}

For a fixed Grothendieck universe $U$, the ($U$-) coproduct completion of $\cat{C}\op$, which we'll call its \emph{menu category} $\mn_U(\cat{C})$, is a very rich construction. Applying it a few times successively to the empty category, one obtains:
\[
	\Cat{1}\cong \mn_U(\Cat{0}),\qquad
	\smset_U\cong\mn_U(\Cat{1}),\qquad
	\poly_U\cong\mn_U(\smset_U).
\]
It is not a 2-monad on $\smcat$, since it technically is of the form
\[
\mn\colon\smcat\co\to\smcat
\]
however, applying it twice does result in a monad, namely the free $U$-distributive category on $\cat{C}$
\[
\fd_U(\cat{C})\cong\mn_U(\mn_U(\cat{C})).
\]
Here $\fd_U(\cat{C})$ has $U$-ary coproducts and $U$-ary products that distribute over them,%
\footnote{The term \emph{distributive category} generally refers to a category where finite products distribute over finite coproducts. In $\fd_U(\cat{C})$, $U$-ary products distribute over $U$-ary coproducts, so it is often called \emph{completely distributive}. We will call it $U$-distributive.}
 and it is the free such thing containing a copy of $\cat{C}$. In this way, we can very roughly think of $\mn$ as a kind of ``square root'' of $\fd$. 

It is well-known that $\poly_U$ is the free $U$-distributive category on a single object, i.e.\ $\poly\cong\fd(\Cat{1})$. By \cite{moss2018dialectica}, we have that the \emph{dependent dialectica category} is given by 
\[\ddial\cong\fd_U(\Cat{2})\]
where $\Cat{2}=(\bullet\to\bullet)$ is the walking arrow category. In this note, we show that much of the abundance of structure one finds in $\poly$ and $\ddial$ in fact arises from this root source.

%---- Section ----%
\section*{Acknowledgments}
We thank Josh Meyers, Nelson Niu, and Brandon Shapiro for many useful conversations.  

This material is based upon work supported by the Air Force Office of Scientific Research under award number FA9550-20-1-0348.


%-------- Chapter --------%
\chapter{The menu construction}

Free completions of categories have been well-studied; in particular, free coproduct completions of categories were a major theme in the recent paper \cite{adamek2020nice}. We have not seen explicit interest in the coproduct completion of the opposite category, but searching for this was difficult and our search may have been inadequate.

For the remainder of the paper, we fix a Grothendieck universe $U$ of sets; it will be implicit. We call a set \emph{small} if it is in $\ol{u}$. The category $\smset$ denotes the category of small sets.

\begin{proposition}
Let $\cat{C}$ be a category. The following categories are equivalent:
\begin{enumerate}
	\item the coproduct completion of $\cat{C}\op$,
	\item the opposite of the product completion of $\cat{C}$, and
	\item the category whose objects are pairs $(A,c)$, where $A\in\smset$ and $c\colon A\to\cat{C}$, and for which
	\begin{align*}
	\Hom((A,c),(A',c'))&\coloneqq\prod_{a\in A}\sum_{a'\in A'}\cat{C}\big(c'(a'),c(a)\big)\\&\cong\left\{(f,\varphi)\;\middle|\; f\in\smset(A,A')\text{ and }\forall a\in A, \varphi(a)\in\cat{C}(c'(fa),c(a))\right\}.
	\end{align*}
\end{enumerate}
We denote this category by $\mn(\cat{C})$.
\end{proposition}
\begin{proof}[Sketch of proof]
We first show that $\mn(\cat{C})$ is a coproduct completion of $\cat{C}$. There is an obvious inclusion $\eta_\cat{C}\colon \cat{C}\op\to\mn(\cat{C})$ sending $c\mapsto (1,c)$. Given a set $I$ and objects $(A_i,c_i)_{i\in I}$, their coproduct $(A,c)\coloneqq\sum_{i\in I}(A_i,c_i)$ is given by
\[
A\coloneqq\sum_{i\in I}A_i
\qqand
c(i,a)\coloneqq c_i(a).
\]
Indeed, there is an inclusion map $(A_i,c_i)\to (A,c)$, given by the coproduct inclusion of sets $A_i\to A$ and identities on the $\cat{C}$-values. Given any other $(B,d)$ and maps $(f_i,\varphi_i)\colon (A_i,c_i)\to(B,d)$, we can define $f\colon A\to B$ by the universal property and $\varphi_{i,a}\coloneqq\varphi_i(a)$ for each $(i,a)\in A$. This makes the appropriate diagrams commute and is unique as such.

Given any other coproduct-complete category $\cat{D}$ equipped with a functor $F\colon\cat{C}\op\to\cat{D}$, there is a unique functor $\mn(\cat{C})\to\cat{D}$ commuting with $\eta$: it sends $(A,c)$ to the coproduct $\sum_{a\in A}F(c_a)$. We leave the details to the reader.

It is clear the opposite of the product completion of $\cat{C}$ is equivalent to the coproduct completion of $\cat{C}\op$. This completes the proof sketch.
\end{proof}

The name \emph{menu category} is perhaps not ideal; we are quite open to it being renamed. The idea is that an object in $\mn(\cat{C})$ consists of a menu of items in $\cat{C}$: one indicates a menu item and looks to receive an object in $\cat{C}$. A morphism $(A,c)\to (A',c')$ is a method for fulfilling the first menu: each menu item $a\in A$ is sent to some menu item $a'\in A'$, as well as a strategy $c'(a')\to c(a)$, meaning that if we receive an element of $c'(a')$, then we receive an element of $c(a)$.%
\footnote{Here we use ``element'' of $c(a)$ in the sense of generalized element: a map into $c(a)$.}






%-------- Chapter --------%
\chapter{Categorical structures on $\mn(\cat{C})$}

\begin{proposition}
Suppose $J$ is a small category. If $\cat{C}$ has colimits of shape $J$, then $\mn(\cat{C})$ has limits of shape $J$. Similarly, if $\cat{C}$ has limits of shape $J$, then $\mn(\cat{C})$ has colimits of shape $J$.
\end{proposition}
\begin{proof}
	The category of sets has $J$-shaped colimits and $J$-shaped limits. Suppose given a diagram $(A,c)\colon J\to \mn(\cat{C})$. Its limit is given by $\lim_{j\in J}(A_j,c_j)\coloneqq(A,c)$, where
	\[
	A\coloneqq\lim_{j\in J}A_j
	\qqand
	c(a)\coloneqq\colim_{j\in J}c_j(\pi_j(a))
	\]
	Here, $\pi_j\colon A\to A_j$ is the projection. It is easy to check that this is a limit.
	
	Colimits are a bit harder. Given the diagram $(A,c)$ as above, its colimit is given by $\colim_{j\in J}(A_j,c_j)\coloneqq(A',c')$, where
	\[
	A'\coloneqq\colim_{j\in J}A_j
	\qqand
	c'(a')\coloneqq
	\]
	**
\end{proof}


%---- Section ----%
\section{Extensivity and distributivity}
\cite{carboni1993introduction}


%---- Section ----%
\section{Monoidal structures}
%---- Section ----%
\section{Closures}
%---- Section ----%
\section{Coclosures}

%-------- Chapter --------%
\chapter{Categorical structures on $\fd(\cat{C})$}
%---- Section ----%
\section{Structures arising from $\fd=\mn\circ\mn$}
%---- Section ----%
\section{The substitution product}

%---- Section ----%
\section{The menu category construction in $\poly$}


\printbibliography
\end{document}
